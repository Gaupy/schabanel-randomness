\documentclass{article}
\usepackage{amsmath}
\usepackage{amsfonts}
\usepackage{amssymb}
\IfFileExists{pdfsync.sty}{\usepackage{pdfsync}}{}
\usepackage[frenchb]{babel}
\usepackage[latin1]{inputenc}
\usepackage{fullpage}


%%% Macros

\newtheorem{define}{Definition}
\newtheorem{theorem}{Theorem}
\newtheorem{corollary}{Corolary}
\newtheorem{proposition}{Proposition}
\newtheorem{remark}{Remark}
\newtheorem{example}{Example}
\newtheorem{lemma}{Lemma}
\newenvironment{proof}{{\bf Proof}~:}{\hfill $\Box$ }
\def\pr{\mathop{\rm Pr}\nolimits}



\begin{document}
\title{Using Randomness in Computer Science}
\author{Nathaniel Carr� \& Mika�l Rabie}
\date{\today}
\maketitle
\section{Self-correction of integer multiplication}
We have the program $A$ (seen as a matrix $N\times N$), and it is supposed to be such that $A(x,y)=xy\mod N$, for $(x,y)\in[0,N-1]^2$. There might be bugs in the matrix (i.e. pairs $(x,y)$ such that $A(x,y)\ne xy\mod N$), but $|\#bugs|\le10\%$ of the pairs. We cannot check the correctness, how can we fix that with randomness ?\\

We can use the following procedure that provides $xy\mod N$ with probability $\ge60\%$
\begin{itemize}
\item Draw 2 random numbers $(r,s)\in[0,N-1]^2$
\item Output $A(x+r,y+s)-A(r,y+s)-A(x+r,s)+A(r,s)$
\end{itemize}

If $A$ provides the good results for our requests, we get :\\
$(x+r)(y+s)-r(y+s)-(x+r)s+rs=xy+xs+ry+rs-ry-rs-xs-rs+rs=xy$.

The access of $A$ asked are uniformly random. To be convinced of that, we just need the following lemma :

\begin{lemma}
  If $G$ is a group and $U$ a uniform random variable over $G$, then $\forall g\in G$, $g+U$ is a uniform random variable over $G$.
\end{lemma}
\begin{proof}
$\forall y\in G$, $\pr\{gU=y\}=\pr\{U=g^{-1}y\}=\frac{1}{|G|}$.

\end{proof}

With that, the probability that each of the entry of A required is wrong is $\le 10\%$. It implies that the output value is not equal to $xy\mod N$ with probability $\le 40\%$. Pr$\{Failure\}\le 40\%$. Because it works for any input, if we repeat this process and take the majority of the value, we can have a probability of success as close to 1 as we want. (We can also compare the result to $A(x,y)$ to improve the program.)\\




Multiplication is "random-self reductible". This problem has a good "worst case" to deal with.

\section{The class PCP$(r,q)$ and GAP-3SAT$_{1,c}$}

\begin{define}
PCP$(r,q)$=$\left\{
\begin{array}{r l}
L| & \exists\text{bPTTM V  st :}\\
 & \text{1 : V uses }\theta(n)\text{ random bits}\\
 & \text{2 : }V(x,y)\text{ probes only }\theta(q)\text{ bits of y}\\
 & \text{3 : if x}\in L,\exists \text{ y st }\pr\{V(x,y)=1\}=1\\
 & \text{4 : if x}\not\in L,\forall \text{ y }\pr\{V(x,y)=1\}\le\frac{1}{2}\\
\end{array}
\right\}$
\end{define}

\begin{define}
SAT=$\left\{
\begin{array}{r l}
\varphi(x)=\bigwedge\limits_{i=j}^mC_j(x)| & \exists \text{x st }\forall j,C_j(x)=1\\
\end{array}
\right\}$

$\varphi=\bigwedge\limits_{i=j}^mC_j\in$SAT iff $\exists x$ satisfying the $m$ clauses.

$\varphi\not\in$SAT iff $\forall x$, $x$ satisfies at most $m-1$ clauses of $\phi$.
\end{define}

In that way, there is a gap between the number of accepted clause by $\varphi$ depending on weather $\varphi$ is in SAT or not.

The gap is :
\begin{itemize}
\item $\ge m$ satisfiable clauses at the same time if $\varphi\in$L.
\item $\le m-1$ satisfiable clauses at the same time if $\varphi\not\in$L.
\end{itemize}

We will now amplify this gap, asking that at most $cm$, with $c<1$, clauses are satisfiable at the same time if $\varphi$ is not totally satisfiable.

\begin{define}
GAP-3SAT$_{1,c}$ is such that :
\begin{itemize}
\item $\varphi\in$GAP-3SAT$_{1,c}$ if $\varphi$ is satisfiable
\item $\varphi\not\in$GAP-3SAT$_{1,c}$ if at most $cn$ clauses of $\varphi$ can be simultaneously satisfied
\end{itemize}
\end{define}

\section{PCP and NP theorem}

We want to prove the theorem :

\begin{theorem}
NP=PCP$(\log n,1)$
\end{theorem}

For that, we will first prove that PCP$(log\,n,\,1)\,\subseteq\,$NP.

Indeed, the verifier choses $r=O(log\,n)$ bits randomly, which means $2^r=Poly(n)$ random choices possible. Thus, the verifier can check all random strings in polynomial time to find whether a word is in the langage or not.
\\

In order to study the property NP$\,\subseteq\,$PCP$(log\,n,\,1)$, we will use the introduced problem GAP-3SAT$_{1,c}$. We have the following property :
\begin{proposition}
NP$ \, \subseteq \, $PCP$(log \, n, \, 1) \, \Longleftrightarrow \, \exists $ $c < 1, \, $GAP-3SAT$_{1,c}$ is NP-hard
\end{proposition}
This proposition is equivalent to :\\
$(\exists c <1,\,\forall \varepsilon > 0)\,\exists\,a\,(c+\varepsilon)-$approximation for MaxSAT $\Longrightarrow $P=NP

\subsubsection{Proof $\Longleftarrow$}

In the proposition 1, the direction $\Longleftarrow$ is the easiest to prove.

Indeed, if GAP-3SAT$_{1,c}$ is a NP-hard problem, then we have a polynomial time reduction. For any language $L \in $NP we have the function $x\mapsto \varphi_{x}$ so that :
\begin{itemize}
\item if $x \in L,\,\varphi_{x}\in $SAT
\item if $x \notin L$, at most $cn$ clauses of $\varphi_{x}$ are satisfied simultaneously
\end{itemize}
$ $

Then, we will create a PCP-verifier which will be $V(x,y)$ where $y=y_{1}...y_{l}$ is the value of the variables of $\varphi_{x}$. The verifier will :
\begin{enumerate}
\item Draw a random clause of $\varphi_{x}$, $C_{j}$
\item Output $C_{j}(y)$
\end{enumerate}
$ $

Indeed, if $x \in L$, $\varphi_{x}$ is satisfiable, so $\exists y$ such that any clause is satisfiable. Then, whatever clause I chose, $C_{j}(y)=1$. If $x \notin L$, at most $cn$ clauses of $\varphi_{x}$ can be satisfied simultaneously so whatever $y$ I chose, there is a constant probability that $C_{j}(y)=0$. $V$ is well defined as a PCP-verifier.

We can check that the parameters are correct, that $L \in $PCP$(log\, n, 1)$ : The clause chosen randomly among a number of clauses polynomial in the size of $x$, so we need $O(log\,|x|)=O(log\,n)$ bits to chose the clause. Then, we use only the value of only $3=O(1)$ variables, in order to calculate the value of the clause. $V$ is a $(log\,n,\,3)$-verifier.

\subsubsection{Proof $\Longrightarrow$}
In the opposite way, we suppose NP$\,\subseteq\,$PCP$(log\,n,\,1)$.

Then, take $L \in NP$ and let $V$ be its $PCP$-verifier. $V$ uses $r=\theta (log\,n)$ random bits, asks for the value of $q=\theta (1)$ positions in y and outputs 0 or 1 such that :
\begin{itemize}
\item If $x \in L,\, \exists y,\, Pr{V(x,y)=1}<1$
\item If $x \notin L,\, \forall y,\, Pr{V(x,y)=0}>1/2$
\end{itemize}
$ $

For a random string $r$ and $q$ positions $y_{1},...,\,y_{q}$ in $y$, we define $f_{r}(y_{1},...,\,y_{q})$ is the output of $V$ for such values. Thus, $f_{r}(y_{1},...,\,y_{q})$ is a $SAT$ formula of $2^q$ clauses, so it is a 3SAT formula of $(q-2)2^q=\theta(1)$. Then we have :
\begin{itemize}
\item If $x \in L,\, \exists y,\, \forall r,\,f_{r}(y_{1},...,\,y_{q})=1$
\item If $x \notin L,\, \forall y,\, \#\lbrace r|f_{r}(y_{1},...,\,y_{q})=1 \rbrace \leq \dfrac{1}{2} 2^r$
\end{itemize}
$ $

Now, let define $\varphi_{x}=\bigwedge_{r}f_{r}(y)$. This formula can be computed in polynomial time, because there is a polynomial number of random string. This formula is of length $2^r(q-2)2^q$. Finally, we can choose $c=1-\dfrac{1}{q2^q}$. Any NP-problem can be reduce to GAP-3SAT$_{1,c}$, which proves the property.
\end{document}
